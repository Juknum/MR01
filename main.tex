%!TEX program = xelatex
\documentclass[12pt, a4paper]{report}
\usepackage[legalpaper, a4paper, margin=2cm]{geometry}
\usepackage[french]{babel}
\usepackage{hyperref}
\usepackage{libs/utbmcovers}
\usepackage{lipsum}
\usepackage{sectsty}
\usepackage{csquotes}
\usepackage[style=ieee]{biblatex}
\usepackage{fancyhdr}

%----------------------------------------
% Use Tahoma as base font
%----------------------------------------

\setmainfont{Tahoma.ttf}[
	Path=./assets/fonts/, 
	Extension=.ttf, 
	BoldFont = *Bold, 
	ItalicFont = *Italic, 
	BoldItalicFont = *BoldItalic,
	BoldSlantedFont = *BoldSlanted,
	SlantedFont = *Slanted,
	SmallCapsFont = *BoldSmallCaps,
]

\chapterfont{\tahomafont\fontseries{b}\fontsize{28pt}{36pt}}
\sectionfont{\tahomafont\fontseries{b}\fontsize{24pt}{28pt}}
\subsectionfont{\tahomafont\fontseries{b}\fontsize{20pt}{22pt}}
\subsubsectionfont{\tahomafont\fontseries{b}\fontsize{20pt}{22pt}}

%----------------------------------------
% UTBM covers configuration
%----------------------------------------

\setutbmfrontillustration{assets/images/utbm_default_illustration}
\setutbmtitle{REPORT TITLE 2}
\setutbmsubtitle{Internship Report ST40 - P2023}
\setutbmstudent{JACQUES Pierre-Paul}
\setutbmstudentdepartment{Computer Science Depertment}
\setutbmstudentpathway{IA - Artificial Intelligence}
\setutbmcompany{Company DEMO-Controlers}
\setutbmcompanyaddress{8 rue de la Fierté\\75 013 Paris}
\setutbmcompanywebsite{www.democontrollers.com}
\setutbmcompanylogo{assets/images/default_company}
\setutbmcompanytutor{COMPANY Tutor}
\setutbmschooltutor{SCHOOL Tutor}
\setutbmkeywords{
	[N°X – Y] Keyword 1,
	[N°X – Y] Keyword 2,
	[N°X – Y] Keyword 3,
	[N°X – Y] Keyword 4,
}
\setutbmabstract{ \lipsum[1-2] }

%----------------------------------------
% Document configuration
% Notes:
% - '\graphicspath' is used to add the path to the images.
% - '\addbibresource' is used to bibliographies files, use comma to add more than one.
%----------------------------------------

\graphicspath{{./assets/images/}}
\addbibresource{bibliography.bib}

%----------------------------------------
% Document
% Notes:
% - Usually, the abstract is not referenced in the table of contents
%	  nor the greetings section, so we use the '*' option to avoid it.
% - Non-cited references are not shown in the bibliography, so we use the
%	  '\nocite{*}' command to show them.
% - Everything below '\subsubsection' is not shown in the table of contents,
%----------------------------------------

\begin{document}
\tahomafont

% \makeutbmfrontcover{}

\topmargin -1.5cm
\pagestyle{fancy}
\fancyhf{}
\fancyhead[L]{
	\textbf{MR01: Méthodologie de recherche}
	\\ Proposition argumentée et problématisé d'un sujet
}
\fancyhead[R]{\thepage}
\renewcommand{\headrulewidth}{1pt}

% présentez votre sujet, votre problématique et de son contexte, 
% justifiez le choix de cette problématique (son intérêt, sa pertinence 
% d’un point de vue sociétal et/ou scientifique) - document word de 1 à 2
% pages ; soyez conscient.e que ce premier document va évoluer et 
% participera par la suite à construire votre introduction 
% - A finaliser avant le 01/10/2023 22:00

\chapter*{MR01: Proposition argumentée et problématisé d'un sujet}
\section*{Sujet: L'accessibilité numérique sur l'image de marque}

\noindent
Aborder un tel sujet va permettre de comprendre et d'examiner comment la 
disponibilité, l'accessibilité et l'expérience utilisateur sur les services et 
resources numérique d'une entité, affectent la perception qu'on les clients de 
cette dernière.
\\\\
L'accessibilité numérique se réfère à la facilité avec laquelle des individus,
y compris ceux ayant des besoins spécifiques, peuvent accéder aux informations,
aux produits ou aux services d'une entité sur internet. Cela passe notamment par
les interfaces utilisateurs, les applications mobiles, les sites web, les 
plateformes en ligne, les réseaux sociaux, etc.
\\\\
Tout type d'entité peut être concerné par l'accessibilité numérique, que ce soit 
une entreprise, une organisation à but non lucratif, une institution 
gouvernementale, etc.
\\\\
\noindent
Ce sujet va permettre une possible réflexion sur les éléments suivants :
\begin{itemize}
	\item \textbf{Accessibilité en ligne}: Comment une entité rend-elle ses
	sites web, applications mobiles et plateformes en ligne accessibles à tous ses 
	utilisateurs, y compris ceux ayant des handicaps visuels, auditifs ou d'autres
	besoins spécifiques ?
	\item \textbf{Expérience utilisateur}: Comment l'accessibilité
	affecte-t-elle l'expérience globale de l'utilisateur ? Comment l'utilisateur
	perçoit-il l'entité en fonction de son expérience avec ces services numériques 
	?
	\item \textbf{Réputation et responsabilité sociale}: L'engagement envers
	l'accessibilité numérique peut contribuer à la réputation d'une entité en 
	tant qu'entité socialement responsable. Comment cela influence-t-il la 
	perception du public ?
	\item \textbf{Conformité aux réglementations}: De nombreuses juridictions 
	exigent désormais de respecter des normes d'accessibilité numérique. Comment 
	la conformité à ces règles influence-t-elle la réputation d'une entité ?
	\item \textbf{Innovation et compétitivité}: Les entités qui investissent dans
	l'accessibilité numérique sont-elles perçues comme étant plus innovantes et
	compétitives dans un monde de plus en plus numérique ?
\end{itemize}

\noindent
Dans un contexte où les services et ressources numériques sont de plus en plus
présents dans notre quotidien, il est important de comprendre comment l'accès à
ces derniers peuvent influencer la perception qu'on les clients d'une entité.
En effet, l'accessibilité numérique est un enjeu majeur pour toute entité qui 
souhaite étendre son audience et ses bénéfices.
\\\\
Par conséquent la problématique suivante est soulevée : \textbf{Comment
l'accessibilité numérique influence-t-elle l'image de marque d'une entité ?}

% \section*{Greetings}
% \lipsum[1-2] 

% \tableofcontents
% \newpage

% \chapter{Lorem Ipsum}
% \section{Section of Lorem Ipsum}
% \lipsum[1-1]\footnote{Footnote of Lorem Ipsum}

% \subsection{Sub Section of Lorem Ipsum}
% \lipsum[1-1]

% \subsubsection{Sub Sub Section of Lorem Ipsum}
% \paragraph{title of paragraph}
% \lipsum[1-1]

% \paragraph{}
% \lipsum[1-1]

% \newpage

% \nocite{*}
% \printbibliography{}

% \makeutbmbackcover{}
\end{document}